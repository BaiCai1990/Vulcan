\documentclass{article}

\usepackage[top=1in,left=1in,right=1in,bottom=1in]{geometry}
\usepackage{hyperref}
\usepackage{enumitem}

\setlist[description]{leftmargin=2\parindent,labelindent=2\parindent} % description is used for fields, indent to 
                                                                    % it easier to identify them

\title{Vulcan LCM Message Interface}
\author{Collin Johnson}

\begin{document}

\maketitle
\tableofcontents

\section{Introduction}

This document describes the LCM interface for controlling Vulcan. The messages are divided into categories based on the 
module they interact with.

Note that the LCM interface is a subset of the full Vulcan functionality, but allows for interoperability between 
different programming languages.

\section{Local Metric HSSH Commands}

\subsection{Inputs:}

\subsubsection{save\_lpm\_command}

Tell the local\_metric\_hssh module to save the current LPM it has constructed to a file.

\paragraph{Channel:} 

SAVE\_LPM

\paragraph{Fields:}

\begin{description}
 \item[filename:] The filename to save the LPM to.
\end{description}


\paragraph{Effects:}

\begin{itemize}
  \item An LPM will be saved in a file with the provided filename. This LPM is saved in the LPM's internal file format, 
which is based on the cereal serialization library. The file can only be read from a C++ program.
\end{itemize}


\paragraph{Errors:}

\begin{itemize}
  \item If the file cannot be created, then no map will be saved. A message will be output to the local\_metric\_hssh 
console indicating the failure.
\end{itemize}


\subsubsection{load\_lpm\_command}

Tell the local\_metric\_hssh module to load an LPM and attempt to relocalize within it. A guess at the position of the 
robot in the LPM must be provided to assist in relocalizing in the map.

A quirk with the current relocalization approach requires the robot to be moving slowly to help it relocalize. After 
sending the command, drive the wheelchair slowly and it should relocalize.

Relocalization is typically very fast as long as initial position estimate is good and clutter in the environment hasn't 
changed much since the LPM being loaded was saved. Relocalization will occasionally fail by using the wrong 
orientation. Currently, the only way to see if a failure has occurred is to look at the robot pose estimate in the 
DebugUI. If relocalization fails, just try sending the message again. Relocalization is almost always succeeded within 
two or three attempts.

\paragraph{Channel:} 

LOAD\_LPM

\paragraph{Fields:}

\begin{description}
 \item[filename:] Filename containing the LPM to load.
 \item[initial\_x:] x-position of center of region in which robot is located in the LPM being loaded.
 \item[initial\_y:] y-position of center of region in which robot is located in the LPM being loaded.
\end{description}


\paragraph{Effects:}

\begin{itemize}
  \item An LPM will be loaded from a file with the provided filename. This file must have been saved in the LPM 
file format, which is always true if saved via save\_lpm\_command.
  \item The robot will begin attempting to relocalize in the LPM specified in the command. The relocalization guess 
will be a square with side length 2m centered at (initial\_x, initial\_y).
  \item After a successful relocalization, the robot will begin using the loaded LPM for local SLAM or localization.
\end{itemize}

\paragraph{Errors:}

\begin{itemize}
  \item If the LPM file cannot be opened, then relocalization will not be attempted. The only way to detect this 
failure is to look at the console running the local\_metric\_hssh module.
  \item Relocalization can fail if:
    \begin{itemize}
      \item the environment is ambiguous.
      \item the environment doesn't sufficiently match the saved LPM due to clutter.
      \item the particle filter has bad luck sampling the correct pose.
      \item the initial position guess is too far from the robot's actual position.
    \end{itemize}
      
  \item Relocalization failure is easiest to see in the DebugUI, where the robot will not be facing the correct 
direction in the map
\end{itemize}


\subsection{Outputs:}

\subsubsection{robot\_pose\_t}

Sent every time the localization is updated in local\_metric\_hssh. The pose is the current $(x,y,\theta)$ estimate of 
the robot's location in the LPM.

\paragraph{Channel:} 

ROBOT\_POSE

\paragraph{Fields:}

\begin{description}
    \item[timestamp:] Time at which this pose estimate was calculated.
    \item[x:] x-position of the robot in the LPM's frame of reference.
    \item[y:] y-position of the robot in the LPM's frame of reference.
    \item[theta:] orientation of the robot in the LPM's frame of reference.
\end{description}

\paragraph{Effects:}
N/A

\paragraph{Errors:}
N/A

\section{Metric Planner Interface}

\subsection{Inputs:}

\subsubsection{metric\_pose\_target\_t}

Tell the metric\_planner module to drive to the specified pose $(x,y,\theta)$ or position $(x,y)$.

There's also a simple program that is installed in the bin/ folder to test sending the message. In build/bin/ you can 
run:

\begin{verbatim}
  ./metric\_pose\_target\_test x y theta
\end{verbatim}

to command the robot to drive to $(x,y,\theta)$. The $\theta$ is optional. If you don't provide it, then a position 
target will be sent.

\paragraph{Channel:}

METRIC\_POSE\_TARGET

\paragraph{Fields:}

\begin{description}
  \item[id:] Unique identifier for the target to identify the target in status messages
  \item[x:] x-position to drive to in current LPM.
  \item[y:] y-position to drive to in current LPM.
  \item[theta:] desired orientation of the robot at the $(x,y)$ target (only if not a position target).
  \item[is\_position:] Flag indicating if it only matters if the robot drives to a position $(x,y)$ rather than a 
pose $(x, y, \theta)$. If $is_position = 1$, the robot will just drive to $(x,y)$ as efficiently as possible, rather 
than trying to also reach the target orientation
\end{description}

\paragraph{Effects:}

\begin{itemize}
  \item The robot will begin planning and navigating a trajectory to the specified target position or pose.
  \item Once planning begins, metric\_pose\_target\_status\_t messages will be sent for each update of the planner.
  \item This command will override any existing commands being executed by the metric\_planner.
\end{itemize}

\paragraph{Errors:}

\begin{itemize}
  \item If the desired target cannot be safely reached by the robot, then it will fail to get there. The exact cause of 
an execution failure will be reported in a metric\_pose\_target\_status\_t message.
\end{itemize}


\subsection{Outputs:}

\subsubsection{metric\_pose\_target\_status\_t}

Sent every time the metric\_planner is updated while planning and driving a route to a metric\_pose\_target\_t.

\paragraph{Channel:} 

METRIC\_POSE\_TARGET\_STATUS

\paragraph{Fields:}

\begin{description}
    \item[timestamp:] Time at which the message was generated.
    \item[target\_id:] Target for which this status applies.
    \item[status:] Status of the planner in reaching the target, see below constants for possible values.
\end{description}

\hfill
\\
The status is an enumerated value providing information on the state of execution for the planner.
\begin{description}
    \item[IDLE:] The robot doesn't currently have a task.
    \item[IN\_PROGRESS:] The robot is currently driving towards the target
    \item[REACHED\_TARGET:] The robot reached the target it was driving to.
    \item[FAILED\_CANNOT\_ASSIGN\_TARGET:] The robot cannot go to the target because it is too close to a wall or the 
target is in a never-been-visited part of the map.
    \item[FAILED\_ROBOT\_IS\_STUCK:] The robot cannot move for some reason.
    \item[FAILED\_CANNOT\_FIND\_SOLUTION:] No path to the target exists.
\end{description}

\paragraph{Effects:}
N/A

\paragraph{Errors:}
N/A

\section{Control Planner Commands}

\subsection{Inputs:}

\subsubsection{direct\_control\_command}

Command the control\_planner to drive using a simple set of commands:

\begin{description}
  \item[Stop:] Robot will stop where it is
  \item[Go straight:] The robot sets a target 3m in front of it and starts driving there. When it gets within 1m, it 
sets a new target 3m in front of it, this repeats until it can't move forward anymore. The behavior is a little finicky 
if lots of clutter is around, but works well enough in a hallway. The robot will only drive to within 0.5m of a wall 
before stopping.
  \item[Turn left/right:] The robot will attempt to turn around. If the robot isn't able to turn-in-place, it will arc 
away from the nearest obstacle until it can turn-in-place. This means the wheelchair might drive forward a little before 
actually starting to turn. The robot motion can be a little slow and jerky when turning in place.
\end{description}


You can test the different motions using the \emph{keyboard\_control} program in build/bin. When you start it, it will 
give the keys to use. Just enter the appropriate letter, hit enter, and it will send the direct\_control\_command to 
the control\_planner.

\paragraph{Channel:}

DIRECT\_CONTROL\_COMMAND

\paragraph{Fields:}

\begin{description}
 \item[timestamp:] Used for diagnostics, but doesn't affect handling of the command.
 \item[source:] Informational data providing the source of the command message. The control\_planner outputs the 
source of the message, but doesn't otherwise use it.
  \item[command:] you know what to do with this one.
  \item[speed:] Specify the desired speed at which to execute the command. If driving forward, it is the meters per 
second to drive. If rotating, it is the radians per second to turn. This option isn't currently supported in MPEPC, so 
it doesn't do anything
\end{description}

\paragraph{Effects:}

\begin{itemize}
  \item The robot will attempt to execute the specified command using the behavior described above. As long as the 
robot can safely perform the requested motion, it will do so.
  \item This command will override any existing commands being executed by the metric\_planner.
\end{itemize}

\paragraph{Errors:}

\begin{itemize}
 \item If the robot isn't able to execute the specified command, then it won't move. Not moving is not necessarily 
indicative of failure, it just means the robot cannot safely execute the command.
\end{itemize}

\end{document}
