\documentclass{article}

\usepackage[top=1in,bottom=1in,left=1in,right=1in]{geometry}

\title{HSSH Requirements and Design}
\author{Collin Johnson}

\begin{document}

\maketitle

\section{Local Metric}

The Local Metric layer of the HSSH has two purposes:

\begin{enumerate}
 \item Provide situational awareness sufficient for safe local navigation.
 \item Provide a stable, high-resolution map of the local surround for determining the local topological structure of 
  the environment.
\end{enumerate}

To accomplish (1), the LPM maintained by the Local Metric HSSH must keep the current target pose within the map. As 
long as the robot is driving towards the target, this requirement will always be satisfied.

Accomplishing (2) requires more care to ensure the Local Topo HSSH doesn't have an artificially difficult task to 
perform. First, the LPM needs to grow to fit the entirety of the current area being navigated by the robot. Doing so 
keeps the area labeling from losing information. Once an area has been exited, the LPM can be truncated to remove the 
area and prevent accidental loop closures in the LPM.

Additionally, the Local Topo HSSH works better if the LPM axes are aligned with the environment. Thus, a request to 
reorient the map can be made. For the rotation, the map rotates in-place by a requested amount, preserving cells as 
much as possible given blurring due to cells not moving by an integer number of cells.

Note, this rotation doesn't change the reference frame of the LPM, it just changes the cell-to-world mapping. By not 
changing the reference frame, the tracked objects, waypoints, areas, etc. aren't invalidated, they just fall into 
different cells on the next update. In order to ensure that an implicit dependence on the orthogonality of the 
environment doesn't sneak into the topological map, the lambda for the path segment records the rotation amount for use 
in the relaxation of the path segments used for finding the global layout.

The following messages support the required functionality for controlling behavior of the Local Metric HSSH from the 
other parts of the HSSH and MPEPC:

\begin{description}
 \item[Set SLAM Mode:] \hfill \\
    \begin{description}
     \item[Mapping Mode:] The map can be set to either \emph{scroll} to keep the robot within a fixed radius of the 
        center of the map while maintaining a fixed size or to \emph{expand} as the robot explores new portions of 
        the map, allowing the robot to drive right up to the edge of the map even.
     \item[Localization Mode:] Set the Local Metric HSSH to only localize within the current LPM. The map won't be 
        updated. This mode is most useful for testing functionality of the motion planner or when operating in a large 
        metric map of a known environment.
     \item[High-Resolution Mode:] The mapping can create a high-resolution version of the LPM. Switching to the 
        pre-determined resolution, either 1cm or 2cm cells instead of 5cm cells.
    \end{description}

 \item[Save Map:] Saves the current LPM -- and glass map if available -- to the requested file.
 
 \item[Relocalize In LPM:] Request the robot attempt to relocalize within the provided LPM or from an LPM loaded from a 
    specified file.
 \item[Rotate LPM:] Request the map be rotated around its center by the specified number of radians.
 \item[Truncate LPM:] Request the map be truncated to the contain only the portion within the provided boundary 
    rectangle.
\end{description}


\section{Local Topological (Topo)}

The Local Topo HSSH layer has two purposes:

\begin{enumerate}
 \item Segment and label the LPM into distinct areas.
 \item Determine when topological events (area transitions or turning around) have occurred.
\end{enumerate}

Local Topo HSSH requires Local Metric HSSH to be running in stretch-mode. A SetMappingMode command is sent at 1Hz to 
change the mode to stretch-mode.

\subsection{Response to Topological Events}

Area transitions and turning around on a path are the topological events created by the Local Topo HSSH.

\paragraph{Area Transition:}

When the robot leaves an area, the LPM needs to be adjusted to chop out the part of the map containing the old area. 
This action is taken to ensure that no loops are accidentally closed outside of Global Topo HSSH. The LPM is searched, 
starting from the current robot position. A bounding rectangle is drawn around all free cells not falling in the 
exited area. A TruncateMap command is sent to Local Metric HSSH to truncate the LPM to include only this portion of 
the map, which may, due to it being a potentially loose-fitting bounding rectangle, overlap some or all of the area 
trying to be removed. However, this bit of overlap shouldn't be a problem.

If the exited area is a path segment, the LPM additionally needs to be rotated to align with the axis of the place that 
was entered. This rotation will be in the opposite direction of $\theta_\lambda$ that was calculated when the 
transition event was detected. This rotation is handled by sending a RotateMap command to Local Metric HSSH with the 
rotation being $-\theta_\lambda$.

\paragraph{Turn Around:}

Nothing special happens when the robot turns around on a path. It simply means it is headed back from whence it came.

\end{document}

